\textbf{f23} & \multicolumn{2}{@{}c@{}}{8.0 \quad} & \multicolumn{2}{@{}c@{}}{193 \quad} & \multicolumn{2}{@{}c@{}}{234 \quad} & \multicolumn{2}{@{}c@{}}{263 \quad} & \multicolumn{2}{@{}c@{}}{299 \quad} & \multicolumn{2}{@{}c@{}}{348 \quad} & \multicolumn{2}{@{}c@{}|}{379} & 15 & /15\\\hline
\algAtables\hspace*{\fill} & \textbf{2} & \textbf{.1}\mbox{\tiny (2)} & \textbf{10} & \textbf{}\mbox{\tiny (9)} & \textbf{1231} & \textbf{}\mbox{\tiny (1090)} & \multicolumn{2}{@{\,}l@{\,}}{\textbf{$\infty$}} & \multicolumn{2}{@{\,}l@{\,}}{\textbf{$\infty$}} & \multicolumn{2}{@{\,}l@{\,}}{\textbf{$\infty$}} & \multicolumn{2}{@{\,}l@{\,}|}{$\infty$\,\textit{2e4}} & 0 & /15\\